\documentclass[12pt]{article}

\usepackage{sbc-template}
\usepackage{graphicx,url}
\usepackage[utf8]{inputenc}
\usepackage[brazil]{babel}
\usepackage{indentfirst}

     
\sloppy

\title{Análise de Desempenho de Caches de Dados e Instruções utilizando Simulador Simplescalar}

\author{Pedro Henrique Warken Ramos }


\address{Universidade Federal de Santa Maria (UFSM)}

\begin{document} 

\maketitle

\section{Introdução} \label{sec:introducao}

Este trabalho tem como objetivo investigar o desempenho de caches de dados e
instruções por meio do uso do simulador Simplescalar, uma ferramenta 
amplamente utilizada para a análise e avaliação de arquiteturas de 
computadores. O estudo foi conduzido com a execução de três aplicações de 
testes providas pelo professor Mateus Beck Rutzig, visando explorar os 
conceitos apresentados na disciplina de Arquitetura de Computadores.

Foram modeladas quatro caches de nível 1, variando-se os parâmetros de
configuração, tais como o número total de conjuntos, o 
tamanho do bloco em bytes, a associatividade e o algoritmo de substituição.
Tais parâmetros foram testados por meio de linha de comando, usando o módulo
"sim-cache" do SimpleScalar, as configurações das caches foram especificadas,
permitindo a execução das aplicações selecionadas. Para este trabalho foi
definido o tamanho padrão das caches como 128 bytes, para que somente a 
organização dos parâmetros influencie nos resultados e não o tamanho total da cache.

As aplicações utilizadas foram selecionadas com o intuito de abranger 
diferentes cenários de uso, contemplando tarefas como multiplicação de 
matrizes, decodificação de arquivos MPEG e cálculos matemáticos simples. 
Durante a execução das aplicações, foram coletados dados relevantes para 
análise, como o número total de instruções executadas, o número de acessos às 
caches, o número de acertos (hits), o número de faltas (misses), o número de 
substituições e o número de gravações em cache.

A partir dos dados coletados, realizamos uma análise comparativa do desempenho 
das quatro caches de dados e instruções. Essa análise foi realizada tanto 
quantitativamente, através da elaboração de gráficos com os resultados 
obtidos, quanto tecnicamente, argumentando sobre os resultados observados e 
relacionando-os com os conceitos apresentados na disciplina de Arquitetura de 
Computadores.

Este trabalho contribui para a compreensão dos principais parâmetros 
de configuração das caches e seu impacto no desempenho do sistema. 
Além disso, fornece uma visão prática sobre a utilização do simulador
Simplescalar como ferramenta de análise e avaliação de arquiteturas de
computadores.

\section{Metodologia} \label{sec:metodologia}
Nesta seção, será descrito em detalhes a metodologia utilizada para realizar a avaliação comparativa de desempenho das caches montadas no Simplescalar. Serão abordados os passos desde a configuração das caches até a coleta e análise dos resultados.

\subsection{Configuração das Caches}
Neste estudo, foi utilizado o simulador Simplescalar "sim-cache" para modelar e simular as caches. Foram configuradas quatro caches de dados e instruções de nível 1, variando os seguintes parâmetros: número total de conjuntos (\textless nsets\textgreater), tamanho do bloco em bytes (\textless bsize\textgreater), associatividade (\textless assoc\textgreater) e algoritmo de substituição (\textless repl\textgreater). Cada cache foi otimizada para uma aplicação específica, enquanto uma cache adicional foi projetada para fornecer a melhor otimização média.

As configurações das caches foram definidas da seguinte forma:

\textbf{Cache 1 (Otimizada para a aplicação "mm.ss"):}
\begin{itemize}
\item Cache de instruções: il1:256:32:4:l
\item Cache de dados: dl1:256:64:8:l
\end{itemize}

\textbf{Cache 2 (Otimizada para a aplicação "amp.ss"):}
\begin{itemize}
\item Cache de instruções: il1:256:32:4:l
\item Cache de dados: dl1:256:64:8:l
\end{itemize}

\textbf{Cache 3 (Otimizada para a aplicação "basicmath.ss"):}
\begin{itemize}
\item Cache de instruções: il1:256:32:4:l
\item Cache de dados: dl1:256:64:8:l
\end{itemize}

\textbf{Cache 4 (Melhor otimização média):}
\begin{itemize}
\item Cache de instruções: il1:256:32:4:l
\item Cache de dados: dl1:256:64:8:l
\end{itemize}

Cada configuração de cache foi projetada para atender às demandas específicas de uma determinada aplicação, considerando o tamanho dos conjuntos, o tamanho do bloco e a associatividade. O algoritmo de substituição utilizado foi o LRU (Least Recently Used) para todas as caches.


\section{Resultados} \label{sec:resultados}

\end{document}

