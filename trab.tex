\documentclass[12pt]{article}

\usepackage{sbc-template}
\usepackage{graphicx,url}
\usepackage[utf8]{inputenc}
\usepackage[brazil]{babel}

     
\sloppy

\title{Análise de Desempenho de Caches de Dados e Instruções utilizando Simulador Simplescalar}

\author{Pedro Henrique Warken Ramos }


\address{Universidade Federal de Santa Maria (UFSM)}

\begin{document} 

\maketitle

\section{Introdução}

Este trabalho tem como objetivo investigar o desempenho de caches de dados e
instruções por meio do uso do simulador Simplescalar, uma ferramenta 
amplamente utilizada para a análise e avaliação de arquiteturas de 
computadores. O estudo foi conduzido com a execução de três aplicações de 
testes providas pelo professor Mateus Beck Rutzig, visando explorar os 
conceitos apresentados na disciplina de Arquitetura de Computadores.

Inicialmente, foram modeladas quatro caches de nível 1, variando-se os 
parâmetros de configuração, tais como o número total de conjuntos, o tamanho 
do bloco em bytes, a associatividade e o algoritmo de substituição. Tais 
parâmetros foram testados por meio de linha de comando, usando o módulo "sim-
cache" do SimpleScalar, as configurações das caches foram especificadas, 
permitindo a execução das aplicações selecionadas.

As aplicações utilizadas foram selecionadas com o intuito de abranger 
diferentes cenários de uso, contemplando tarefas como multiplicação de 
matrizes, decodificação de arquivos MPEG e cálculos matemáticos simples. 
Durante a execução das aplicações, foram coletados dados relevantes para 
análise, como o número total de instruções executadas, o número de acessos às 
caches, o número de acertos (hits), o número de faltas (misses), o número de 
substituições e o número de gravações em cache.

A partir dos dados coletados, realizamos uma análise comparativa do desempenho 
das quatro caches de dados e instruções. Essa análise foi realizada tanto 
quantitativamente, através da elaboração de gráficos com os resultados 
obtidos, quanto tecnicamente, argumentando sobre os resultados observados e 
relacionando-os com os conceitos apresentados na disciplina de Arquitetura de 
Computadores.

Este trabalho contribui para a compreensão dos principais parâmetros de configuração das caches e seu impacto no desempenho do sistema. Além disso, fornece uma visão prática sobre a utilização do simulador Simplescalar como ferramenta de análise e avaliação de arquiteturas de computadores.

\section{First Page} \label{sec:firstpage}

The first page must display the paper title, the name and address of the
authors, the abstract in English and ``resumo'' in Portuguese (``resumos'' are
required only for papers written in Portuguese). The title must be centered
over the whole page, in 16 point boldface font and with 12 points of space
before itself. Author names must be centered in 12 point font, bold, all of
them disposed in the same line, separated by commas and with 12 points of
space after the title. Addresses must be centered in 12 point font, also with
12 points of space after the authors' names. E-mail addresses should be
written using font Courier New, 10 point nominal size, with 6 points of space
before and 6 points of space after.

The abstract and ``resumo'' (if is the case) must be in 12 point Times font,
indented 0.8cm on both sides. The word \textbf{Abstract} and \textbf{Resumo},
should be written in boldface and must precede the text.

\section{CD-ROMs and Printed Proceedings}

In some conferences, the papers are published on CD-ROM while only the
abstract is published in the printed Proceedings. In this case, authors are
invited to prepare two final versions of the paper. One, complete, to be
published on the CD and the other, containing only the first page, with
abstract and ``resumo'' (for papers in Portuguese).

\section{Sections and Paragraphs}

Section titles must be in boldface, 13pt, flush left. There should be an extra
12 pt of space before each title. Section numbering is optional. The first
paragraph of each section should not be indented, while the first lines of
subsequent paragraphs should be indented by 1.27 cm.

\subsection{Subsections}

The subsection titles must be in boldface, 12pt, flush left.

\section{Figures and Captions}\label{sec:figs}


Figure and table captions should be centered if less than one line
(Figure~\ref{fig:exampleFig1}), otherwise justified and indented by 0.8cm on
both margins, as shown in Figure~\ref{fig:exampleFig2}. The caption font must
be Helvetica, 10 point, boldface, with 6 points of space before and after each
caption.

\begin{figure}[ht]
\centering
\includegraphics[width=.5\textwidth]{fig1.jpg}
\caption{A typical figure}
\label{fig:exampleFig1}
\end{figure}

\begin{figure}[ht]
\centering
\includegraphics[width=.3\textwidth]{fig2.jpg}
\caption{This figure is an example of a figure caption taking more than one
  line and justified considering margins mentioned in Section~\ref{sec:figs}.}
\label{fig:exampleFig2}
\end{figure}

In tables, try to avoid the use of colored or shaded backgrounds, and avoid
thick, doubled, or unnecessary framing lines. When reporting empirical data,
do not use more decimal digits than warranted by their precision and
reproducibility. Table caption must be placed before the table (see Table 1)
and the font used must also be Helvetica, 10 point, boldface, with 6 points of
space before and after each caption.

\begin{table}[ht]
\centering
\caption{Variables to be considered on the evaluation of interaction
  techniques}
\label{tab:exTable1}
\includegraphics[width=.7\textwidth]{table.jpg}
\end{table}

\section{Images}

All images and illustrations should be in black-and-white, or gray tones,
excepting for the papers that will be electronically available (on CD-ROMs,
internet, etc.). The image resolution on paper should be about 600 dpi for
black-and-white images, and 150-300 dpi for grayscale images.  Do not include
images with excessive resolution, as they may take hours to print, without any
visible difference in the result. 

\section{References}

Bibliographic references must be unambiguous and uniform.  We recommend giving
the author names references in brackets, e.g. \cite{knuth:84},
\cite{boulic:91}, and \cite{smith:99}.

The references must be listed using 12 point font size, with 6 points of space
before each reference. The first line of each reference should not be
indented, while the subsequent should be indented by 0.5 cm.

\bibliographystyle{sbc}
\bibliography{sbc-template}

\end{document}

